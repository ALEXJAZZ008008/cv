%----------------------------------------------------------------------------------------
%	PACKAGES AND OTHER DOCUMENT CONFIGURATIONS
%----------------------------------------------------------------------------------------

\documentclass{cv} % Use the custom cv.cls style

\usepackage[left=0.75in,top=0.6in,right=0.75in,bottom=0.6in]{geometry} % Document margins
\newcommand{\tab}[1]{\hspace{.2667\textwidth}\rlap{#1}}
\newcommand{\itab}[1]{\hspace{0em}\rlap{#1}}

\name{Alexander C Whitehead} % Your name
\address{72 Birchington Avenue, Huddersfield, HD3~3RB} % Your address
\address{(+44)~07557~985~843 \\ alexndercwhitehead@gmail.com} % Your phone number and email

\begin{document}

%----------------------------------------------------------------------------------------
%	PERSONAL STATEMENT SECTION
%----------------------------------------------------------------------------------------

\begin{rSection}{Personal Statement}

\item During my time studying for my MSc (in Advanced Computer Science) I enrolled on a module titled Image Analysis. I expected this module to focus on image recognition, it did not, this module was my introduction to Medical Imaging. After completing the module I was invited to work on my master’s dissertation in the medical imaging lab, here I worked on motion correction for medical imaging using infrared cameras. This was all the experience I needed to pursue a PhD in the field.
\item In addition to my MSc in Advanced Computer Science I also have a BSc in Computer Science.
\item My PhD focuses mainly on data driven respiratory motion correction for PET/CT. Data driven meaning that the motion of the patient is taken directly from the data from the PET/CT, this is in contrast to my master’s dissertation where the motion was detected using a camera. Motion correction meaning to take the motion that has been detected and attempt to correct for it in the PET/CT data. Respiratory motion is specifically a problem as breathing motion causes a lot of the organs in the abdomen of a patient, which are being imaged, to shift around and blur.

\end{rSection}

%----------------------------------------------------------------------------------------
%	TECHNICAL STRENGTHS SECTION
%----------------------------------------------------------------------------------------

\begin{rSection}{Technical Strengths}

\item C++, C, DirectX, OpenGL, Arduino, C\#, .Net, ASP, Unity, Xamarin, Java, Android, Python, Matlab, Pascal, JavaScript, Git, SVN, BNF,  Flex, Bison, OpenDX, Soldering

\end{rSection}

%----------------------------------------------------------------------------------------
%	EDUCATION SECTION
%----------------------------------------------------------------------------------------

\begin{rSection}{Education}

{\bf Indian Institute of Technology Kanpur} \hfill {\em July 2008 - Present} 
\\ Junior Undergraduate \hfill { Overall GPA: /10}
\\ Department of Chemical Engineering  


\end{rSection}

%----------------------------------------------------------------------------------------
%	EXPERIENCE SECTION
%----------------------------------------------------------------------------------------

\begin{rSection}{Experience}

\begin{rSubsection}{IIT Bombay}{May 2015 - July 2015}{Undergraduate Research}{}
\item Employed CFD software Gerris to carry out 20 simulations and analysed different shapes during oscillation
\item Derived Lamb’s dispersion relation for free oscillation and applied it to calculate strength of different modes
\item Analysed simulated data to obtain velocity field and compared it with theoretical data to obtain results
\end{rSubsection}


%------------------------------------------------

\begin{rSubsection}{IIT Kanpur}{January 2015 - April 2015}{Manufacturing Process Project - Dragon Model}{}
\item Worked in a team of six people and came up with a model of Dragon with movable wings 
\item Designed and fabricated a skeleton model of dragon with movable wings from scratch in lab employing processes of welding, brazing and casting
\item Received Certificate of Appreciation among 40 projects for its artwork and detailing
\end{rSubsection}

\end{rSection}

\end{document}
