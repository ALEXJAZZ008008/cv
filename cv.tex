\documentclass{cv}
\usepackage[left=1cm,top=1cm,right=1cm,bottom=1cm]{geometry}

\name{Alexander C. Whitehead}

\address{(+44)~07557~985~843 \\ alexandercwhitehead@gmail.com}
\address{72 Birchington Avenue, Huddersfield, West Yorkshire, HD3~3RB, UK}

\begin{document}
    \begin{rSection}{Personal Statement}
        I am currently a PhD student at University College London in medical physics. My PhD focuses, in particular, on respiratory motion correction (with the incorporation of motion models) for PET/CT. During my time as a PhD student I have also had an interest in surrogate signal extraction (specifically for dynamic PET) which lead to an exploration of machine learning/neural networks in general. I've applied machine learning to the fields of denoising, reconstruction and more recently arterial input function estimation.
        
        While studying for my PhD I undertook roles at the university teaching electronics, programming and communication. I specifically lead lectures for groups of approximately $200$ students, teaching the fundamentals of programming. Additionally, I have been the primary supervisor of one MSc dissertation and secondary supervisor of a further three. My primary supervision was on a project to apply machine learning to the extraction of respiratory signals from dynamic PET. My secondary supervision were on projects to denoise total body PET data using neural networks and 3D printed phantoms.
        
        Previously, I had studied at the University of Hull for an MSc and BSc in computer science. Here I had a particular interest in computer vision, compiler design and embedded programming. Larger projects I undertook included: A motion correction method for PET, which used a Microsoft Kinect camera to track and correct the motion of the patient in a medical scanner. As well as an online, multiplayer, mobile based game which saw players physically running around cities attempting to cut each other's path off, a la Tron.
        
        While studying for my MSc and BSc I undertook roles at the university which saw me performing research to produce a cave scanning device and an autonomous remote controlled race car. The cave scanning device consisted of a 3D printed custom mounted LiDAR atop a drone, which could fly into a cave and the LiDAR would scan. Furthermore I also taught classes on embedded and distributed programming.
        
        Other than for academic education and experience, I've also enjoyed positions in the past as a bouncer in the University of Hull's nightclub and as a helpful face moving first year students into their accommodation (helping them around the university in general). I was the social secretary of the University of Hull Computer Science Society and also an honorary member of the University of York's Real Ale Society (where I set many unbeaten records). Before Covid struck in full force, I also ran a cake club in the office where I was working on my PhD. Time permitting, I enjoy partaking in music (both attending and performing, I play the saxophone and the bass guitar - previously, I was involved in an outreach program teaching music in primary schools), analogue photography (and film development), 3D printing and beer brewing (I worked at York Brewery for a short time).
    \end{rSection}
    
    \begin{rSection}{Technical Strengths}
        \item Medical Physics, Medical Imaging, Image Registration, Image Analysis, PET/CT, Signal Processing
        \item Data Science, Machine Learning, Deep Learning, Artificial Intelligence, Data Visualisation, Analytical Skills
        \item Computer Graphics, Computer Vision, Virtual Realty
        \item C++, C, C\#, Python, Matlab, Java, JavaScript, Pascal, LaTeX, Shell Scripting, GNU/Linux, Git, SVN
        \item OpenGL, DirectX, Android, Arduino, Xamarin, Raspberry Pi, .NET Framework, Unity, Qt, BNF, EBNF, Flex, SolidWorks, 3D Printing, Soldering 
    \end{rSection}
    
    \begin{rSection}{Education}
        {\bf University College London} \hfill {\em 2018 - Present} 
        \\ PhD Medical Imaging Physics and Medical Image Computing
        
        \item Inverse Problems in Imaging, Information Processing for Medical Imaging, Medical Imaging with Ionising Radiation
        
        \item Project: Improved Quantification for Respiratory Gated PET/CT: Data-Driven Algorithms for Respiratory Motion Correction in PET/CT. This project was multifaceted, initially it began with the development of a respiratory motion correction method for PET/CT. This method was invariant of the position (and hence of the introduced bias) of a misaligned attenuation map. This was achieved through the use of an iterative method, which aligned PET data and moved its reference position to that of the attenuation map. To aid in this, motion models were incorporated, which can be considered as a temporal/gate regulariser. However, the project evolved to include methods to extract respiratory signals from dynamic PET data (for the use of gating and motion model estimation) as well as denoising and reconstruction techniques which use neural networks.
        
        {\bf University of Hull} \hfill {\em 2017 - 2018} 
        \\ MSc Advanced Computer Science \hfill {Distinction}
        
        \item C++ Programming and design, Image Analysis, Real-Time Graphics, Simulation and Artificial Intelligence and Visualisation.
        
        \item Project: Motion Signal Extraction Framework for the Microsoft Kinect Camera: Point Cloud Registration and its Application as a Motion Correction Metric in PET/CT. The goal of this project was the development of a library that could provide motion correction for PET using depth sensing cameras. Deformations were determined by registering depth images (point clouds) before spatially and temporally aligning them with the PET acquisition. These deformations could then be used to correct any motion present.
        
        {\bf University of Hull} \hfill {\em 2014 - 2017} 
        \\ BSc Computer Science \hfill {First Class Honours}
        
        \item Computer Systems, Professional Skills for Computer Science, Programming, Quantitative Methods for Computer Science and Software Engineering and Human Computer Interaction
        
        \item 2D Computer Graphics and User Interface Design, Advanced Programming, Electronics and Interfacing, Networking and Games Architecture, Simulation and 3D Graphics and Systems Analysis, Design and Process.
        
        \item Distributed Systems Programming, Languages and their Compilers, Mobile Devices and Applications and Virtual Environments.
        
        \item Project: Capture the Campus!, this is a GPS based mobile game. The objective is to steer yourself around the periphery of a play area before then travelling across the play area all while avoiding multiple computer-controlled enemies. Once the player has crossed the play area it is then redefined as the larger of the two polygons created by splitting the area in two along the path travelled by the player. A score is awarded to the player based on the size of the polygon removed from the play area. If an enemy intersects the player’s path their path is reset and they lose some score. A local multiplayer session can be instantiated on one phone and joined by others running the application.
    \end{rSection}
    
    \begin{rSection}{Experience}
        \begin{rSubsection}{University College London}{2018 - Present}{Senior Postgraduate Teaching Assistant}{}
            \item I currently work at University College London. At University College London I work as a senior postgraduate teaching assistant. This role involves the teaching of and other assistance with the running of multiple modules across many academic years. For instance, this role included the lecturing of over 200 students in the fundamentals of programming.
        \end{rSubsection}
        
        \begin{rSubsection}{University of Hull}{2017-2018}{Demonstrator}{}
            \item I worked at the University of Hull from the summer of 2017 until the summer of 2018. At the University of Hull I worked as a Demonstrator. This role involved aiding with the teaching of certain modules within the university. For instance, this role included leading embedded programming labs.
        \end{rSubsection}
        
        \begin{rSubsection}{University of Hull (Digital Centre)}{2017}{Research Intern}{}
            \item I worked at the University of Hull (Digital Centre) on a temporary basis during the summer of 2017. At the University of Hull (Digital Centre) I worked as a Research Intern. This role involved researching virtual reality through room scanning devices, autonomous vehicles, 3D printing and the internet of things. For instance, this role included the development of a drone mounted scanning device as well as a miniature autonomous vehicle intended for racing.
        \end{rSubsection}
        
        \begin{rSubsection}{DREAMTHINKSPEAK}{2017}{VR Assistant}{}
            \item I worked at DREAMTHINKSPEAK on a temporary basis during the summer of 2017. At DREAMTHINKSPEAK I worked as a VR Assistant. This role involved the development and deployment of a virtual reality art exhibit based on Korean apartments for the city of culture celebration.
        \end{rSubsection}
    \end{rSection}
    
    \begin{rSection}{Publications}
        \begin{rSubsection}{Data Driven Surrogate Signal Extraction for Dynamic PET Using Selective PCA: An Ensemble Method}{Submitted 2022}{IOP Physics in Medicine and Biology}{}
            \item {\bf Whitehead~A.C.}, Su~K.-H., Emond~E.C., Biguri~A., Machado~M., Porter~J.C., Garthwaite~H., Wollenweber~S.D., McClelland~J.R. \& Thielemans~K.
        \end{rSubsection}
        
        \begin{rSubsection}{PET/CT Motion Correction Exploiting Motion Models Fit on Coarsely Gated Data Applied to Finely Gated Data}{2022}{IEEE NSS-MIC}{}
            \item {\bf Whitehead~A.C.}, Su~K.-H., Wollenweber~S.D., McClelland~J.R. \& Thielemans~K.
        \end{rSubsection}
        
        \begin{rSubsection}{Data Driven Surrogate Signal Extraction for Dynamic PET Using Selective PCA}{2022}{IEEE NSS-MIC}{}
            \item {\bf Whitehead~A.C.}, Su~K.-H., Emond~E.C., Biguri~A., Machado~M., Porter~J.C., Garthwaite~H., Wollenweber~S.D., McClelland~J.R. \& Thielemans~K.
        \end{rSubsection}
        
        \begin{rSubsection}{Pseudo-Bayesian DIP Denoising as a Preprocessing Step for Kinetic Modelling in Dynamic \newline PET}{2022}{IEEE NSS-MIC}{}
            \item {\bf Whitehead~A.C.}, Erlandsson~K., Biguri~A., Wollenweber~S.D., McClelland~J.R. \& Thielemans~K.
        \end{rSubsection}
        
        \begin{rSubsection}{Physically Informed Neural Network for Non-Invasive Arterial Input Function Estimation In Dynamic PET Imaging}{2022}{MIDL}{}
            \item Ferrante~M., Inglese~M., Brusaferri~L., {\bf Whitehead~A.C.}, Loggia~M. \& Toschi~N.
        \end{rSubsection}
        
        \begin{rSubsection}{Detection Efficiency Modelling and Joint Activity and Attenuation Reconstruction in non-TOF 3D PET from Multiple-Energy Window Data}{2021}{IEEE Transactions on Radiation and Plasma Medical Sciences}{}
            \item Brusaferri~L., Emond~E.C., Bousse~A., Twyman~R., {\bf Whitehead~A.C.}, Atkinson~D., Ourselin~S., Hutton~B.F., Arridge~S. \& Thielemans~K.
        \end{rSubsection}
        
        \begin{rSubsection}{Comparison of Motion Correction Methods Incorporating Motion Modelling for PET/CT Using a Single Breath Hold Attenuation Map}{2021}{IEEE NSS-MIC}{}
            \item {\bf Whitehead~A.C.}, Biguri~A., Su~K.-H., Wollenweber~S.D., Stearns~C.W., Hutton~B.F., McClelland~J.R. \& Thielemans~K.
        \end{rSubsection}
        
        \begin{rSubsection}{Systematic evaluation of the impact of involuntary motion in whole body dynamic PET}{2021}{IEEE NSS-MIC}{}
            \item Biguri~A., Kotasidis~F., {\bf Whitehead~A.C.}, Burger~I., Hutton~B.F., \& Thielemans~K.
        \end{rSubsection}
        
        \begin{rSubsection}{Respiratory Motion Correction With a Single Attenuation Map Using NAC Derived Deformation Fields}{2020}{IEEE NSS-MIC}{}
            \item {\bf Whitehead~A.C.}, Biguri~A., Efthimiou~N., Su~K.-H., Wollenweber~S.D., Stearns~C.W., Hutton~B.F., McClelland~J.R. \& Thielemans~K.
        \end{rSubsection}
        
        \begin{rSubsection}{Impact of Time-of-Flight on Respiratory Motion Modelling using Non-Attenuation-Corrected \newline PET}{2019}{IEEE NSS-MIC}{}
            \item {\bf Whitehead~A.C.}, Emond~E.C., Efthimiou~N., Akintonde~A., Wollenweber~S., Stearns~C.W., Hutton~B.F., McClelland~J.R. \& Thielemans~K.
        \end{rSubsection}
        
        \begin{rSubsection}{ Preliminary investigation of the impact of Axial Ring Splitting on Image Quality for the Cost Reduction of Total-Body PET}{2019}{IEEE NSS-MIC}{}
            \item Efthimiou~N., {\bf Whitehead~A.C.}, Stockhoff~M., Thyssen~C., Archibald~S.J. \& Vandenberghe~S.
        \end{rSubsection}
    \end{rSection}
    
    \begin{rSection}{Code}
        \begin{rSubsection}{CCP SyneRBI SIRF}{2021}{Zenodo}{}
            \item Ovtchinnikov~E., Brown~R., Mayer~J., Pasca~E., da~Costa-Luis~C.O., Atkinson~D., Kolbitsch~C., Efthimiou~N., Gillman~A., Biguri~A., Deidda~D., Ehrhardt~M.J., {\bf Whitehead~A.C.}, Fardell~G., Thomas~B.A., Leek~F., \& Thielemans~K.
        \end{rSubsection}
        
        \begin{rSubsection}{SIRF Virtual Machine}{2021}{Zenodo}{}
            \item Ovtchinnikov~E., Brown~R., Thielemans~K., Pasca~E., da~Costa-Luis~C.O., Thomas~B.A., Atkinson~D., Mayer~J., Gillman~A., Kolbitsch~C., Ehrhardt~M.J. \& {\bf Whitehead~A.C.}
        \end{rSubsection}
        
        \begin{rSubsection}{CCPPETMR SIRF}{2019}{Zenodo}{}
            \item Ovtchinnikov~E., Brown~R., Pasca~E., da~Costa-Luis~C.O., Thomas~B.A., Atkinson~D., Mayer~J., Gillman~A., Kolbitsch~C., Ehrhardt~M.J., {\bf Whitehead~A.C.}, \& Thielemans~K.
        \end{rSubsection}
    \end{rSection}
    
    \begin{rSection}{References}
        Available upon request.
    \end{rSection}
\end{document}
