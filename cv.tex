\documentclass{cv}
\usepackage[left=1cm,top=1cm,right=1cm,bottom=1cm]{geometry}

\linespread{1.1}

\name{Alexander~C.~Whitehead}

\address{(+44)7557~985~843 $\oplus$ alexandercwhitehead@gmail.com}
\address{Fuhlsbuettler Str.~599, 22337 Hamburg, Germany}

\begin{document}
    \begin{rSection}{Experience}
        \item \begin{rSubsection}{London South Bank University}{2025 - Present}{Visiting Fellow}{}
            \item \begin{itemize}
                \item Led a project to develop a neural network-based approach for extracting metabolite-corrected arterial input functions from dynamic PET data.
                \item Contributed to the selection and mentoring of PhD students.
            \end{itemize}
        \end{rSubsection}
        
        \item \begin{rSubsection}{University College London}{2022 - 2024}{Research Fellow}{}
            \item \begin{itemize}
                \item Led multiple projects in collaboration with the Open Source Imaging Consortium (OSIC).
                \item Contributed to a project using neural network-based methods to predict survival duration for patients with IPF using CT scans.
                \item Designed an advanced explainable technique, surpassing Grad-CAM, to generate lung maps highlighting regions contributing to decreased survival likelihood.
                \item Developed a diffusion-based method to synthesize future CT scans, aiding disease progression forecasting and clinical planning.
            \end{itemize}
        \end{rSubsection}
        
        \item \begin{rSubsection}{University College London}{2018 - 2024}{Senior Postgraduate Teaching Assistant}{}
            \item \begin{itemize}
                \item Assisted in the teaching, examination, and administration of multiple modules across several academic years.
                \item Delivered lectures on programming fundamentals to over 200 students.
                \item Led small-group workshops on mathematics and provided hands-on support in electrical engineering labs.
            \end{itemize}
        \end{rSubsection}
        
        \item \begin{rSubsection}{University of Hull}{2017 - 2018}{Demonstrator}{}
            \item \begin{itemize}
                \item Assisted in the teaching of multiple modules, providing support in lectures and practical sessions.
                \item Contributed to embedded programming labs, guiding students through hands-on applications and problem-solving.
            \end{itemize}
        \end{rSubsection}
        
        \item \begin{rSubsection}{University of Hull (Digital Centre)}{2017}{Research Intern}{}
            \item \begin{itemize}
                \item Conducted research on virtual reality, room scanning technologies, autonomous vehicles, 3D printing, and the Internet of Things.
                \item Developed a drone-mounted scanning device for 3D environment reconstruction.
                \item Designed a miniature autonomous vehicle for high-speed racing applications.
            \end{itemize}
        \end{rSubsection}
    \end{rSection}
    
    \begin{rSection}{Education}
        \item \begin{rSubsection}{University College London}{2018 - 2025}{PhD Medical Imaging Physics and Medical Image Computing}{}            
            \item \textbf{Research Project:} \textit{Improved Quantification for Respiratory Gated PET/CT: Data-Driven Algorithms for Respiratory Motion Correction in PET/CT}
            
            \item \begin{itemize}
                \item Developed a respiratory motion correction method for PET/CT, designed to be invariant to the position and bias introduced by misaligned attenuation maps.
                \item Implemented an iterative approach aligning PET data to the attenuation map’s reference position, incorporating motion models as temporal/gate regularisers.
                \item Expanded the project to include methods for extracting respiratory signals from dynamic PET data for gating and motion model estimation.
                \item Investigated denoising and reconstruction techniques using neural networks to enhance PET/CT image quality and quantification.
            \end{itemize}
        \end{rSubsection}
        
        \item \begin{rSubsection}{University of Hull}{2017 - 2018}{MSc Advanced Computer Science}{Distinction}
            \item \textbf{Dissertation Project:} \textit{Motion Signal Extraction Framework for the Microsoft Kinect Camera: Point Cloud Registration and its Application as a Motion Correction Metric in PET/CT}
            
            \item \begin{itemize}
                \item Developed a software library for motion correction in PET imaging using depth-sensing cameras.
                \item Implemented point cloud registration techniques to track deformations and align them spatially with PET acquisitions.
                \item Worked towards applying these deformations to correct motion artefacts in PET scans, enhancing image accuracy and quantification.
            \end{itemize}
        \end{rSubsection}
        
        \item \begin{rSubsection}{University of Hull}{2014 - 2017}{BSc Computer Science}{First Class Honours}
            \item \textbf{Dissertation Project:} \textit{Capture the Campus!} – A GPS-based, networked multiplayer mobile game inspired by Qix.
            
            \item \begin{itemize}
                \item Developed a real-world interactive gameplay experience where players physically move across a map, tracked via GPS.
                \item Implemented real-time networking to enable local multiplayer sessions, allowing players to compete in a dynamic digital environment.
                \item Designed and developed the game for mobile platforms, incorporating real-world movement as a core gameplay mechanic.
            \end{itemize}
        \end{rSubsection}
    \end{rSection}

    \begin{rSection}{Technical Strengths}
        \item \begin{itemize}
            \item \textbf{Medical Imaging and Physics:} PET/CT, Image Registration, Image Analysis, Signal Processing, Medical Imaging with Ionising Radiation.
            \item \textbf{Data Science and AI:} Machine Learning, Deep Learning, Artificial Intelligence, Data Visualisation, Analytical Skills.
            \item \textbf{Computer Graphics and Vision:} Computer Vision, Virtual Reality, 2D and 3D Graphics, Image Processing.
            \item \textbf{Programming Languages:} C++, C, C\#, Python, MATLAB, Java, JavaScript, Pascal, LaTeX, Shell Scripting.
            \item \textbf{Software and Tools:} GNU/Linux, Git, SVN, TensorFlow, PyTorch, scikit-learn, NumPy, OpenGL, DirectX, Android, Unity, Qt, .NET Framework, BNF, EBNF, Flex.
            \item \textbf{Hardware and Engineering:} Arduino, Raspberry Pi, 3D Printing, SolidWorks, Electronics, Soldering.
        \end{itemize}
    \end{rSection}
    
    \begin{rSection}{Languages}
        \item \begin{itemize}
            \item \textbf{English:} Native
            \item \textbf{German:} A1
        \end{itemize} 
    \end{rSection}

    \begin{rSection}{References}
        \item \begin{rSubsection}{Ludovica Brusaferri}{Long-term Collaborator}{Senior Lecturer, London South Bank University}{}
            \item \textbf{Email:} ludovica.brusaferri@lsbu.ac.uk
        \end{rSubsection}
        
        \item \begin{rSubsection}{Kris Thielemans}{PhD Supervisor}{Professor, University College London}{}
            \item \textbf{Email:} k.thielemans@ucl.ac.uk
        \end{rSubsection}

        \item \begin{rSubsection}{Jamie McClelland}{PhD Supervisor}{Associate Professor, University College London}{}
            \item \textbf{Email:} j.mcclelland@ucl.ac.uk
        \end{rSubsection}
    \end{rSection}
    
    \begin{rSection}{}
        \item Kind regards,
        \\
        \\
        \\
        \\
        \\
        \item Alexander C. Whitehead
    \end{rSection}

    \newpage

    \begin{rSection}{Personal Statement}
        \item I am a researcher with expertise in medical imaging, machine learning, and computational modelling, particularly in the application of neural networks for medical data analysis. My most recent role was as a Research Fellow at University College London (UCL) focusing on principled machine learning research. I was funded by OSIC (Open Source Imaging Consortium), giving me access to a vast database of IPF (Idiopathic Pulmonary Fibrosis) CT scans and clinical data. My primary contributions included contributing to CenTime, a novel censoring technique that improves upon traditional survival analysis methods like Cox and DeepHit. I extended CenTime to generate mortality maps, highlighting the lung regions most associated with patient survival. Additionally, I developed generative models that predict future disease progression in CT scans. Throughout this role, I regularly presented my work at OSIC general meetings attended by leading experts in the field.

        \item My PhD at UCL focused on respiratory motion correction for PET/CT, integrating motion models to improve quantification accuracy. This research expanded into areas such as surrogate signal extraction for dynamic PET, image denoising, and neural network-based reconstruction techniques. My PhD involved close collaboration with clinical staff at the university hospital, and I actively engaged with the wider research community by frequently presenting at conferences, often with multiple contributions per event.

        \item Alongside my research, I was heavily involved in teaching and supervision. As a Senior Postgraduate Teaching Assistant, I delivered lectures to over 200 students on programming fundamentals, led smaller mathematics workshops, and assisted in electrical engineering labs. I also supervised one MSc dissertation on machine learning for respiratory signal extraction in PET and co-supervised three additional MSc projects focused on denoising total-body PET data and 3D-printed phantoms.

        \item Prior to my PhD, I earned an MSc and BSc in Computer Science from the University of Hull, where I developed a strong foundation in computer vision, compiler design, and embedded programming. My MSc research involved using a Microsoft Kinect camera for motion correction in PET imaging, while my BSc project was a GPS-based, networked multiplayer mobile game inspired by Qix and Tron, where players physically moved through real-world environments.

        \item During my time at Hull, I also worked on autonomous vehicle research and cave scanning technology, contributing to a project that aimed to develop a drone-mounted LiDAR system for 3D environment mapping. Additionally, I taught embedded and distributed programming as a demonstrator.

        \item Beyond my academic career, I have pursued a variety of interests. I assisted first-year students to move in and adjust to university, and served as Social Secretary of the Computer Science Society. Before the pandemic, I organised a cake club in my PhD office. In my spare time I enjoy, playing saxophone and bass guitar, analogue/digital photography and film development, 3D printing, and fish-keeping.
    \end{rSection}

    \newpage
    
    \begin{rSection}{Publications}
        \item \begin{rSubsection}{Mortality Maps: Inherently Interpretable Event-Conditional Survival Analysis}{2025 (in review)}{arXiv}{}
            \item \textbf{Whitehead~A.C.} and Brusaferri~L. \\
        \end{rSubsection}
        
        \item \begin{rSubsection}{Advancing Generalisable Neural Network-Based PET Quantification: A Multicenter [11C]PBR28 study}{2025}{IEEE EMBC}{}
            \item Brusaferri~L., \textbf{Whitehead~A.C.}, Maccioni~L., Alshelh~Z., Veronese~M., Turkheimer~F., Toschi~N., Ferrante~M., Inglese~M., Catana~C., Grisan~E., and Loggia~M.L. \\
        \end{rSubsection}
        
        \item \begin{rSubsection}{Data Driven Surrogate Signal Extraction for Dynamic PET Using Selective PCA: An Ensemble Method}{2024}{Physics in Medicine and Biology}{}
            \item \textbf{Whitehead~A.C.}, Su~K.-H., Emond~E.C., Biguri~A., Machado~M., Porter~J.C., Garthwaite~H., Wollenweber~S.D., McClelland~J.R., and Thielemans~K. \\

            \item \textbf{Selected as Editor’s Choice article of the month!} \\
        \end{rSubsection}
        
        \item \begin{rSubsection}{CenTime: Event-conditional modelling of censoring in survival analysis}{2024}{Medical Image Analysis}{}
            \item Shahin~A.H, Zhao~A., \textbf{Whitehead~A.C.}, Alexander~D.C., Jacob~J., and Barber D.
        \end{rSubsection}

        \item \begin{rSubsection}{Physically informed deep neural networks for metabolite-corrected plasma input function estimation in dynamic PET imaging}{2024}{Computer Methods and Programs in Biomedicine}{}
            \item Ferrante~M., Inglese~M., Brusaferri~L., \textbf{Whitehead~A.C.}, Maccioni~L., Turkheimer~F.E., Nettis~M.A., Mondelli~V., Howes~O., Loggia~M.L., Veronese~M., and Toschi~N.
        \end{rSubsection}

        \item \begin{rSubsection}{A Bayesian Neural Network-Based Method For The Extraction Of A Metabolite Corrected Arterial Input Function From Dynamic [$^{11}$C]PBR28 PET}{2023}{IEEE NSS MIC RTSD}{}
            \item \textbf{Whitehead~A.C.}, Brusaferri~L., Maccioni~L., Ferrante~M., Inglese~M., Alshelh~Z., Veronese~M., Toschi~N., Gilman~J., Thielemans~K., and Loggia~M.L.
        \end{rSubsection}

        \item \begin{rSubsection}{Neural Network Based Methods for the Survival Analysis of Idiopathic Pulmonary Fibrosis Patients from a Baseline CT Acquisition}{2023}{IEEE NSS MIC RTSD}{}
            \item \textbf{Whitehead~A.C.}, Shahin~A.H., Zhao~A., Alexander~D.C., Jacob~J., and Barber~D.
        \end{rSubsection}
        
        \item \begin{rSubsection}{PET/CT Motion Correction Exploiting Motion Models Fit on Coarsely Gated Data Applied to Finely Gated Data}{2022}{IEEE NSS MIC RTSD}{}
            \item \textbf{Whitehead~A.C.}, Su~K.-H., Wollenweber~S.D., McClelland~J.R., and Thielemans~K.
        \end{rSubsection}
        
        \item \begin{rSubsection}{Data Driven Surrogate Signal Extraction for Dynamic PET Using Selective PCA}{2022}{IEEE NSS MIC RTSD}{}
            \item \textbf{Whitehead~A.C.}, Su~K.-H., Emond~E.C., Biguri~A., Machado~M., Porter~J.C., Garthwaite~H., Wollenweber~S.D., McClelland~J.R., and Thielemans~K.
        \end{rSubsection}
        
        \item \begin{rSubsection}{Pseudo-Bayesian DIP Denoising as a Preprocessing Step for Kinetic Modelling in Dynamic \newline PET}{2022}{IEEE NSS MIC RTSD}{}
            \item \textbf{Whitehead~A.C.}, Erlandsson~K., Biguri~A., Wollenweber~S.D., McClelland~J.R., and Thielemans~K.
        \end{rSubsection}
        
        \item \begin{rSubsection}{Physically Informed Neural Network for Non-Invasive Arterial Input Function Estimation In Dynamic PET Imaging}{2022}{MIDL}{}
            \item Ferrante~M., Inglese~M., Brusaferri~L., \textbf{Whitehead~A.C.}, Loggia~M., and Toschi~N.
        \end{rSubsection}
        
        \item \begin{rSubsection}{Detection Efficiency Modelling and Joint Activity and Attenuation Reconstruction in non-TOF 3D PET from Multiple-Energy Window Data}{2021}{IEEE Transactions on Radiation and Plasma Medical Sciences}{}
            \item Brusaferri~L., Emond~E.C., Bousse~A., Twyman~R., \textbf{Whitehead~A.C.}, Atkinson~D., Ourselin~S., Hutton~B.F., Arridge~S., and Thielemans~K.
        \end{rSubsection}
        
        \item \begin{rSubsection}{Comparison of Motion Correction Methods Incorporating Motion Modelling for PET/CT Using a Single Breath Hold Attenuation Map}{2021}{IEEE NSS MIC RTSD}{}
            \item \textbf{Whitehead~A.C.}, Biguri~A., Su~K.-H., Wollenweber~S.D., Stearns~C.W., Hutton~B.F., McClelland~J.R., and Thielemans~K.
        \end{rSubsection}
        
        \item \begin{rSubsection}{Systematic evaluation of the impact of involuntary motion in whole body dynamic PET}{2021}{IEEE NSS MIC RTSD}{}
            \item Biguri~A., Kotasidis~F., \textbf{Whitehead~A.C.}, Burger~I., Hutton~B.F., and Thielemans~K.
        \end{rSubsection}
        
        \item \begin{rSubsection}{Respiratory Motion Correction With a Single Attenuation Map Using NAC Derived Deformation Fields}{2020}{IEEE NSS MIC RTSD}{}
            \item \textbf{Whitehead~A.C.}, Biguri~A., Efthimiou~N., Su~K.-H., Wollenweber~S.D., Stearns~C.W., Hutton~B.F., McClelland~J.R., and Thielemans~K.
        \end{rSubsection}
        
        \item \begin{rSubsection}{Impact of Time-of-Flight on Respiratory Motion Modelling using Non-Attenuation-Corrected \newline PET}{2019}{IEEE NSS MIC RTSD}{}
            \item \textbf{Whitehead~A.C.}, Emond~E.C., Efthimiou~N., Akintonde~A., Wollenweber~S., Stearns~C.W., Hutton~B.F., McClelland~J.R., and Thielemans~K.
        \end{rSubsection}
        
        \item \begin{rSubsection}{Preliminary investigation of the impact of Axial Ring Splitting on Image Quality for the Cost Reduction of Total-Body PET}{2019}{IEEE NSS MIC RTSD}{}
            \item Efthimiou~N., \textbf{Whitehead~A.C.}, Stockhoff~M., Thyssen~C., Archibald~S.J., and Vandenberghe~S.
        \end{rSubsection}
    \end{rSection}
    
    \begin{rSection}{Code}
        \item \begin{rSubsection}{STIR Software for Tomographic Image Reconstruction}{2024}{Zenodo}{}
            \item Efthimiou~N., Mustafovic~S., Brown~R., Twyman Skelly~R., Deidda~D., Tsoumpas~C., Falcon~C., Jehl~M., Strugari~M., Khateri~P., Beisel~T., Wadhwa~P., Borgeaud~T., Emond~E., Jacobson~M., Gillman~A., Zverovich~A., Fuster Marti~B., Labbe~C., Biguri~A., Fischer~J., Roethlisberger~M., Bertolli~O., Brusaferri~L., Pasca~E., Thomas~B.A, Aguiar~P., Niknejad~T., Sadki~M., Schmidtlein~C.R., Kerrouche~N., Dikaios~N., Fardell~G., Ehrhardt~M., Valente~P., Ovtchinnikov~E., Schramm~G., Völgyes~D., Dinelle~K., Belluzzo~D., Jurjew~N., Ching~D., Hague~D., Tunnicliffe~H., Chen~G., Porter~S.D., Mikhaylova~E., Dao~V.A., da Costa-Luis~C.O., \textbf{Whitehead~A.C.}, Rashidnasab~A., Gillen~R., Vavrek~J., Tsai~Y.-J., Kohr~H., tokkot, El Katib~M., and Thielemans~K.
        \end{rSubsection}
        
        \item \begin{rSubsection}{SIRF Synergistic Image Reconstruction Framework}{2023}{Zenodo}{}
            \item Ovtchinnikov~E., Brown~R., Mayer~J., Pasca~E., da Costa-Luis~C.O., Atkinson~D., Kolbitsch~C., Efthimiou~N., Strugari~M., Gillman~A., Porter~S.D., Biguri~A., Deidda~D., \textbf{Whitehead~A.C.}, Papoutsellis~E., Fardell~G., Thomas~B.A., Leek~F., Ehrhardt~M., and Thielemans~K.
        \end{rSubsection}

        \item \begin{rSubsection}{SIRF-SuperBuild}{2023}{Zenodo}{}
            \item Pasca~E., Thomas~B.A., da Costa-Luis~C.O., Brown~R., Murgatroyd~L., Biguri~A., Atkinson~D., Ovtchinnikov~E., Gillman~A., Kolbitsch~C., Mayer~J., \textbf{Whitehead~A.C.}, Schramm~G., and Thielemans~K.
        \end{rSubsection}
        
        \item \begin{rSubsection}{SIRF Virtual Machine}{2023}{Zenodo}{}
            \item Ovtchinnikov~E., Brown~R., Thielemans~K., Pasca~E., da Costa-Luis~C.O., Thomas~B.A., Atkinson~D., Mayer~J., Gillman~A., Kolbitsch~C., Ehrhardt~M., and \textbf{Whitehead~A.C.}
        \end{rSubsection}
    \end{rSection}
\end{document}
